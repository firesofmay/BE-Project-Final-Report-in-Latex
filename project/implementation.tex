\chapter{Research Methodology}

\section{Passive Information Gathering Module}

\subsection{HTTP}
To implement Web Scraping module for Information Gathering we need to understand the HTTP protocol. HTTP (The Hypertext Transfer 
Protocol) is an application protocol for distributed, collaborative, hypermedia information systems. HTTP is the 
foundation of data communication for the World Wide Web.

\subsubsection{Request methods}
HTTP defines nine methods (sometimes referred to as "verbs") indicating the desired action to be performed on the identified 
resource. What this resource represents, whether pre-existing data or data that is generated dynamically, depends on the 
implementation of the server. Often, the resource corresponds to a file or the output of an executable residing on the server.\\[0.5cm]
Following are the main request methods we will be dealing with :-
\begin{description}
\item[POST : ] Submits data to be processed (e.g., from an HTML form) to the identified resource. The data is included in the body of the request. This may result in the creation of a new resource or the updates of existing resources or both.
\item[GET : ] Requests a representation of the specified resource. Requests using GET should only retrieve data and should have no other 
effect. (This is also true of some other HTTP methods.) The W3C has published guidance principles on this distinction, saying, 
"Web application design should be informed by the above principles, but also by the relevant limitations."
\end{description}

\subsection{urllib2}
Since we will be using Python 2.x for our project, we will be using the standard library for doing the HTTP requests to be made.
The urllib2 module defines functions and classes which help in opening URLs (mostly HTTP) in a complex world — basic and digest authentication, redirections, cookies and more.\cite{py_urllib2}

urllib2.urlopen(url[, data][, timeout])\\[0.5cm]

\emph{ - Open the URL url, which can be either a string or a Request object.}

\subsection{Regular Expressions}
A Regular Expression is the term used to describe a codified method of searching invented, or defined, by the American mathematician Stephen Kleene.\cite{whatisregex}