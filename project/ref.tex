\addcontentsline{toc}{chapter}{References}
\begin{thebibliography}{99}

% \bibitem{TEXT} is how you refer to your reference in your report. keep that very short
% \emph{Paper Name} is just to highligh the paper name by making it italic, not required but looks nice

%Example
%\bibitem{short_paper_name}\emph{Paper Name}; Author Name, Conference Name, year etc 

\bibitem{paper_honeybot}\emph{Honeybot, Your Man in the Middle for Automated Social Engineering};Tobias Lauinger, Veikko Pankakoski, Davide Balzarotti, Engin Kirda; EURECOM Sophia-Antipolis, France. LEET’10 Proceedings of the 3rd USENIX conference on Large-scale exploits and emergent threats.

\bibitem{paper_allyourcontacts}\emph{All Your Contacts Are Belong to Us: Automated Identity Theft Attacks on Social Networks}; Leyla Bilge, Thorsten Strufe, Davide Balzarotti, Engin Kirda EURECOM Sophia Antipolis, France. WWW ’09 Proceedings of the 18th international conference on World Wide Web.

\bibitem{paper_eightfriends}\emph{Eight Friends Are Enough Social Graph Approximation via Public Listings}; Joseph Bonneau, Jonathan Anderson, Frank Stajano, Ross Anderson. SNS ’09: Proceedings of the Second ACM EuroSys Workshop on Social Network Systems.

\bibitem{paper_towardsautomating}\emph{Towards Automating Social Engineering Using Social Networking Sites};Huber, M.; Kowalski, S.; Nohlberg, M.; Tjoa, S.; Computational Science and Engineering, 2009. CSE ’09. International Conference.

\bibitem{book_se}Social Engineering : The art of Human Hacking


\bibitem{link_humanweak}Link Name \\ \url{http://goliath.ecnext.com/coms2/gi_0199-7186209/}
\url{The-human-element-the-weakest}


\bibitem{marketshare}Market Share of Social Network Sites \\ \url{http://techcrunch.com/2011/12/22/googlesplus}
\bibitem{wiki_scraping}Web Scraping \\ \url{http://en.wikipedia.org/wiki/Web_scraping}

\bibitem{wiki_xmpp}Extensible Messaging and Presence Protocol\\ \url{http://en.wikipedia.org/wiki/Extensible_Messaging_and_Presence_Protocol}

\bibitem{py_urllib2}Urllib2 Python Library\\ \url{http://docs.python.org/library/urllib2.html}

\bibitem{whatisregex}Regular Expression\\ \url{http://www.zytrax.com/tech/web/regex.html}

\bibitem{xmpppy}XMPPPY Python Library\\ \url{http://xmpppy.sourceforge.net/}

\bibitem{xmpppy_issue1}Issues using XMPPPY Link 1\\ \url{http://stackoverflow.com/questions/4732230/xmpppy-and-facebook-chat-integration}

\bibitem{xmpppy_issue2}Issues using XMPPPY Link 2\\ \url{http://stackoverflow.com/questions/8841367/using-xmpp-for-facebook-chat-in-python}

\bibitem{xmpppy_issue3}Issue Resolved using XMPPY\\ \url{http://superuser.com/questions/387504/unable-to-connect-to-facebook-chat-via-python-using-xmpppy-library}

\bibitem{sleekxmpp}SleekXMPP Python Library\\ \url{http://sleekxmpp.com/}

\bibitem{wiki_megahal}MegaHAL\\ \url{http://en.wikipedia.org/wiki/MegaHAL}



%And how you would refer to the paper in the text - example
%Your text in other tex files \cite{short_paper_name} contd explaining

%If you have to display links, do it via  adding \\ (newline) otherwise it goes off the margins. Don't know how to fix this yet
%\bibitem{short_link_name}Link Name \\ \url{http://url link here}
%\url{The-human-element-the-weakest}



\end{thebibliography}