\chapter{Literature Survey}

%First paper
%Example :
%\section[Towards AES using SNS]{Towards Automating Social Engineering Using Social Networking Sites}
%Note short names are optional, for it to show shorter name in table of contents if its too long.
\section{Towards Automating Social Engineering Using Social Networking Sites}

\subsection{Summary}

In this paper, we saw the use of artificial intelligence to create a chat bot
to do the automatic social engineering attack via social networking sites.
Experiments were done on a small group of people on facebook.


\subsection{Advantages}

%This is how you write things in bullet points
\begin{itemize}
\item{Automatic tool, minimal input required from the user.}
\item{If perfected, could be the best way to automate social engineering attack.}

\end{itemize}

\subsection{Disadvantages}

%This is how you write things in bullet points
\begin{itemize}
\item{Fails to convince that the chat bot is a real human.}
\item{Has no input of real world information.}
\item{Chat algorithm is hard coded using regular expressions.}
\item{No real world testing because of ethical issues.}
\item{Building a real world chat bot is out of scope.}

\end{itemize}

%Add Second paper

\section{All Your Contacts Are Belong to Us: Automated Identity Theft Attacks on Social Networks}
\subsection{Summary}

Presents a novel concept of profile hijacking and cloning, which laid the
foundation for the proposed idea.

\subsection{Advantages}
%This is how you write things in bullet points
\begin{itemize}
\item{Works with all social networking sites which has no authentication mechanism of who you say is who you are in real world.}
\item{Very Easy to do, as most of the data required is already available.}

\end{itemize}

\subsection{Disadvantages}
%This is how you write things in bullet points
\begin{itemize}
\item{Does not give us information which is not available directly.}
\item{Impersonating or Faking a human identity is a punishable act.}
\item{Real world testing has ethical issues.}

\end{itemize}


%Add Third Paper
\section{Honeybot, Your Man in the Middle for Automated Social Engineering}
\subsection{Summary}

This paper presents the basic idea we can do man in the middle attack on two
users. The testing was done on public IRC’s but was suggested that it can
be done by using profile hijacking on a social networking sites like Facebook.

\subsection{Advantages}
%This is how you write things in bullet points
\begin{itemize}
\item{Results of success is much higher than by using artificial intelligence via a chat bot.}
\item{Easy to build compared to building artificial intelligence.}

\end{itemize}

\subsection{Disadvantages}
%This is how you write things in bullet points
\begin{itemize}
\item{Required gender conversion in IRC’s chat when impersonating a different gender.}
\item{Testing was done on IRC’s which is not targetted information gathering.}

\end{itemize}

%Add Fourth Paper


\section{Eight Friends Are Enough Social Graph Approximation via Public Listings}
\subsection{Summary}

Presents a novel idea of Publically available data of users on social networking
sites. Also presents the idea of Dominating Sets. It was also mentioned that
the facebook friends public data policy keeps changing, before it was 10
friends than it became 8 and now again its back to 10.


\subsection{Advantages}
%This is how you write things in bullet points
\begin{itemize}
\item{Avoids detection by Facebook.}
\item{Presents a novel idea of using Dominating Sets.}

\end{itemize}

\subsection{Disadvantages}
%This is how you write things in bullet points
\begin{itemize}
\item{Is not succesful in the current scenario where users are security concious.}
\item{Has very low sucess rate as per our experiments.}

\end{itemize}